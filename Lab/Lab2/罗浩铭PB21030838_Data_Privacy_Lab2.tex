
\documentclass[twoside,11pt]{article}
%\documentclass[UTF8]{ctexart}
\usepackage[heading=true]{ctex}

\usepackage{fancyhdr} % 页眉页脚
\usepackage{graphicx}
\usepackage{amsmath}
\usepackage[colorlinks=true, allcolors=blue]{hyperref}
%\usepackage[margin=1.5in]{geometry}

\oddsidemargin .25in    %   Note \oddsidemargin = \evensidemargin
\evensidemargin .25in
\marginparwidth 0.07 true in
%\marginparwidth 0.75 true in
%\topmargin 0 true pt           % Nominal distance from top of page to top of
%\topmargin 0.125in
\topmargin -0.1in
\addtolength{\headsep}{0.25in}
\textheight 8.5 true in       % Height of text (including footnotes & figures)
\textwidth 6.0 true in        % Width of text line.
\widowpenalty=10000
\clubpenalty=10000


\pagestyle{fancy}

%\firstpageno{1}

\title{Data\_Privacy\_Lab2}

\author{罗浩铭\ PB21030838}


\begin{document}

\fancyhf{} % 清除所有页眉页脚
\fancyfoot[C]{\thepage} % 设置右页脚为页码
\fancyhead[l]{\footnotesize USTC Data Privacy}
% 设置右页眉为章节标题 

\renewcommand{\headrulewidth}{0pt} % 去页眉线

\begin{center}
    \textbf{\LARGE{Data\ Privacy\ Lab2}}\\
    \vspace{0.1cm}
    \large{罗浩铭\ PB21030838}
\end{center}


% 实验报告要求:说明代码实现方法,简要给出实验结果说明,可以证明有效性即可

\section{Permutation Cipher}
% (50`)基于 paillier 同态加密实现 VFL-LR 算法,保护训练中间变量,避免产生隐私泄露。
%  补全模型训练过程中的前向及反向传播的具体代码,记录 cancer 数据集在训练过程中的loss及acc变化。


% (20`)请说明代码中 scale 函数的原理及作用。


% (20`)当前代码在每个 epoch 开始时使用 epoch 值作为随机数种子,请说明含义,并实现另一种方式以达到相同的目的。


% (10`)开放题:试分析VFL-LR训练流程中潜在的隐私泄露风险,并简要说明可能的保护方式
\end{document}
