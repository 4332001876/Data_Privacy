
\documentclass[twoside,11pt]{article}
%\documentclass[UTF8]{ctexart}
\usepackage[heading=true]{ctex}

\usepackage{fancyhdr} % 页眉页脚
\usepackage{graphicx}
\usepackage{amsmath}
\usepackage[colorlinks=true, allcolors=blue]{hyperref}
%\usepackage[margin=1.5in]{geometry}

\oddsidemargin .25in    %   Note \oddsidemargin = \evensidemargin
\evensidemargin .25in
\marginparwidth 0.07 true in
%\marginparwidth 0.75 true in
%\topmargin 0 true pt           % Nominal distance from top of page to top of
%\topmargin 0.125in
\topmargin -0.1in
\addtolength{\headsep}{0.25in}
\textheight 8.5 true in       % Height of text (including footnotes & figures)
\textwidth 6.0 true in        % Width of text line.
\widowpenalty=10000
\clubpenalty=10000


\pagestyle{fancy}

%\firstpageno{1}

\title{数据隐私HW3}

\author{罗浩铭\ PB21030838}


\begin{document}

\fancyhf{} % 清除所有页眉页脚
\fancyfoot[C]{\thepage} % 设置右页脚为页码
\fancyhead[l]{\footnotesize USTC Data Privacy}
% 设置右页眉为章节标题 

\renewcommand{\headrulewidth}{0pt} % 去页眉线

\begin{center}
    \textbf{\LARGE{数据隐私HW3}}\\
    \vspace{0.1cm}
    \large{罗浩铭\ PB21030838}
\end{center}
\vspace{0.2cm}



\vspace{0.2cm}
% ============1
\section{Permutation Cipher}
\subsection*{(a)}

\begin{align*}
    \pi^{-1}=
    \begin{pmatrix}
        1 & 2 & 3 & 4 & 5 & 6 & 7 & 8 \\
        4 & 1 & 6 & 2 & 7 & 3 & 8 & 5
    \end{pmatrix}^{-1}
    =
    \begin{pmatrix}
        4 & 1 & 6 & 2 & 7 & 3 & 8 & 5 \\
        1 & 2 & 3 & 4 & 5 & 6 & 7 & 8
    \end{pmatrix} \\
    =
    \begin{pmatrix}
        1 & 2 & 3 & 4 & 5 & 6 & 7 & 8 \\
        2 & 4 & 6 & 1 & 8 & 3 & 5 & 7
    \end{pmatrix}
\end{align*}

\subsection*{(b)}
原始密文如下:

TGEEMNEL NNTDROEO AAHDOETC SHAEIRLM

将其按照每8个字母一组分组,得到:

TGEEMNEL NNTDROEO AAHDOETC SHAEIRLM

将每组按照(a)中的逆置换规则进行逆置换,得到:

GENTLEME NDONOTRE ADEACHOT HERSMAIL

则解密得到明文为:
GENTLEMEN DO NOT READ EACH OTHER'S MAIL

\section{Perfect Secrecy}
\subsection*{(a)}
证明如下:

因为$|\mathcal{M}| = |\mathcal{C}| = |\mathcal{K}|$;对于每个$k \in \mathcal{K}$,其被选中的概率相等;
对于每个$m \in \mathcal{M}$与每个$c \in \mathcal{C}$,都有且仅有一个$k \in \mathcal{K}$使得$e_k(m) = c$:

\begin{tabular}{|c|c|c|}
    \hline
    $m$ & $c$ & $k$ \\
    \hline
    1   & 1   & 1   \\
    1   & 2   & 3   \\
    1   & 3   & 2   \\
    2   & 1   & 2   \\
    2   & 2   & 1   \\
    2   & 3   & 3   \\
    3   & 1   & 3   \\
    3   & 2   & 2   \\
    3   & 3   & 1   \\
    \hline
\end{tabular}

由此,由香农定理可得,该密码系统是完美安全的。

\subsection*{(b)}


\end{document}