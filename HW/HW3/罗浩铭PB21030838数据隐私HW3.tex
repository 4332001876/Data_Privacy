
\documentclass[twoside,11pt]{article}
%\documentclass[UTF8]{ctexart}
\usepackage[heading=true]{ctex}

\usepackage{fancyhdr} % 页眉页脚
\usepackage{graphicx}
\usepackage{amsmath}
\usepackage[colorlinks=true, allcolors=blue]{hyperref}
%\usepackage[margin=1.5in]{geometry}

\oddsidemargin .25in    %   Note \oddsidemargin = \evensidemargin
\evensidemargin .25in
\marginparwidth 0.07 true in
%\marginparwidth 0.75 true in
%\topmargin 0 true pt           % Nominal distance from top of page to top of
%\topmargin 0.125in
\topmargin -0.1in
\addtolength{\headsep}{0.25in}
\textheight 8.5 true in       % Height of text (including footnotes & figures)
\textwidth 6.0 true in        % Width of text line.
\widowpenalty=10000
\clubpenalty=10000


\pagestyle{fancy}

%\firstpageno{1}

\title{数据隐私HW3}

\author{罗浩铭\ PB21030838}


\begin{document}

\fancyhf{} % 清除所有页眉页脚
\fancyfoot[C]{\thepage} % 设置右页脚为页码
\fancyhead[l]{\footnotesize USTC Data Privacy}
% 设置右页眉为章节标题 

\renewcommand{\headrulewidth}{0pt} % 去页眉线

\begin{center}
    \textbf{\LARGE{数据隐私HW3}}\\
    \vspace{0.1cm}
    \large{罗浩铭\ PB21030838}
\end{center}

% ============1
\section{Permutation Cipher}
\subsection*{(a)}

\begin{align*}
    \pi^{-1}=
    \begin{pmatrix}
        1 & 2 & 3 & 4 & 5 & 6 & 7 & 8 \\
        4 & 1 & 6 & 2 & 7 & 3 & 8 & 5
    \end{pmatrix}^{-1}
    =
    \begin{pmatrix}
        4 & 1 & 6 & 2 & 7 & 3 & 8 & 5 \\
        1 & 2 & 3 & 4 & 5 & 6 & 7 & 8
    \end{pmatrix} \\
    =
    \begin{pmatrix}
        1 & 2 & 3 & 4 & 5 & 6 & 7 & 8 \\
        2 & 4 & 6 & 1 & 8 & 3 & 5 & 7
    \end{pmatrix}
\end{align*}

\subsection*{(b)}
原始密文如下:

TGEEMNEL NNTDROEO AAHDOETC SHAEIRLM

将其按照每8个字母一组分组,得到:

TGEEMNEL NNTDROEO AAHDOETC SHAEIRLM

将每组按照(a)中的逆置换规则进行逆置换,得到:

GENTLEME NDONOTRE ADEACHOT HERSMAIL

则解密得到明文为:
GENTLEMEN DO NOT READ EACH OTHER'S MAIL

\section{Perfect Secrecy}
\subsection*{(a)}
证明如下:

因为$|\mathcal{M}| = |\mathcal{C}| = |\mathcal{K}|$;对于每个$k \in \mathcal{K}$,其被选中的概率相等;
对于每个$m \in \mathcal{M}$与每个$c \in \mathcal{C}$,都有且仅有一个$k \in \mathcal{K}$使得$e_k(m) = c$:
\begin{center}
    \begin{tabular}{|c|c|c|}
        \hline
        $m$ & $c$ & $k$ \\
        \hline
        1   & 1   & 1   \\
        1   & 2   & 3   \\
        1   & 3   & 2   \\
        2   & 1   & 2   \\
        2   & 2   & 1   \\
        2   & 3   & 3   \\
        3   & 1   & 3   \\
        3   & 2   & 2   \\
        3   & 3   & 1   \\
        \hline
    \end{tabular}
\end{center}

由此,由香农定理可得,该密码系统是完美安全的。

\subsection*{(b)}
因为该密码系统是完美安全的,且$|\mathcal{M}| = |\mathcal{C}| = |\mathcal{K}|$,由香农定理得:
\begin{itemize}
    \item $\forall k \in \mathcal{K}, P(k)=\frac{1}{|\mathcal{K}|}$
    \item $\forall m \in \mathcal{M}, c \in \mathcal{C}$,有且仅有一个$k \in \mathcal{K}$使得$e_k(m) = c$
\end{itemize}

对于每一个$m \in \mathcal{M}$,固定$m$,有以下结论:

因为$\forall c \in \mathcal{C}$,有且仅有一个$k \in \mathcal{K}$使得$e_k(m) = c$,
所以对于满足$e_k(m) = c$的密文密钥对,其构成映射$f: \mathcal{C} \to \mathcal{K}$。

因为同一个key加密$m$之后得到的密文是唯一的,所以$\forall c_1, c_2 \in \mathcal{C} s.t. c_1 \ne c_2$,其加密密钥不同,
则$f: \mathcal{C} \to \mathcal{K}$为单射。

又由于对于每个$k \in \mathcal{K}$,均存在$c \in \mathcal{C}$使得$e_k(m) = c$,所以$f: \mathcal{C} \to \mathcal{K}$为满射。

综上,$f: \mathcal{C} \to \mathcal{K}$为双射。

则$\forall c \in \mathcal{C}, P(c)=P(f(c))=\frac{1}{|\mathcal{K}|}=\frac{1}{|\mathcal{C}|}$

则对于每一个$m \in \mathcal{M}$,$\mathcal{C}$中每个密文出现的概率相等。

则$\mathcal{C}$中每个密文出现的概率相等。


\section{RSA}
\subsection*{(a)}
因为$p=101,q=113$,
所以$n=pq=11413$,$\phi(n)=(p-1)(q-1)=11200$。

由于$\phi(n)=2^6 5^2 7^1$,则$\phi(\phi(n))=11200=11200 \frac{1}{2} \frac{4}{5} \frac{6}{7} = 3840$

公钥中$n$只有一种,$e$要与$\phi(n)$互质,且$1<e<\phi(n)$,所以$e$有$\phi(\phi(n))-1=3839$种。

则公钥有$3839$种。

\subsection*{(b)}
Bob收到的密文为$C = M^e \pmod{n} = 9726^{3533} \pmod{11413} = 5761$。

因为$e=3533$,所以$ed \equiv 1 \pmod{\phi(n)}$,即$3533d \equiv 1 \pmod{11200}$,使用扩展欧几里得算法解得$d=6597$。

Bob用自己的私钥$d$将$C$其解密得到明文为$M = C^d \pmod{n} = 7624^{6597} \pmod{11413} = 9726$。

\subsection*{(c)}
对于不同的质数$p,q$,$n=pq$,$\phi(n)=(p-1)(q-1)$

则$n-\phi(n)+1=p+q$。

由韦达定理可得,$p,q$为$x^2 - (n-\phi(n)+1)x + n =0$的两个根。

则可得$p, q$的值为$\frac{1}{2} [(n-\phi(n)+1) \pm \sqrt{(n-\phi(n)+1)^2 - 4n}]$算出的两个值($p,q$必须取不同值)。

上述公式的计算只需用到$n$和$\phi(n)$的值,且计算时间复杂度为$O(1)$。因此,若知道$n$和$\phi(n)$的值,则可以在多项式时间内计算出$p,q$的值。

\section{Multi-Party Computation}
\subsection*{(a)}

\subsection*{(b)}




\section{Computational Security}
\subsection*{(a)}

\subsection*{(b)}


\end{document}