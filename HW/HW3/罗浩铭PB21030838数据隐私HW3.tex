
\documentclass[twoside,11pt]{article}
%\documentclass[UTF8]{ctexart}
\usepackage[heading=true]{ctex}

\usepackage{fancyhdr} % 页眉页脚
\usepackage{graphicx}
\usepackage{amsmath}
\usepackage[colorlinks=true, allcolors=blue]{hyperref}
%\usepackage[margin=1.5in]{geometry}

\oddsidemargin .25in    %   Note \oddsidemargin = \evensidemargin
\evensidemargin .25in
\marginparwidth 0.07 true in
%\marginparwidth 0.75 true in
%\topmargin 0 true pt           % Nominal distance from top of page to top of
%\topmargin 0.125in
\topmargin -0.1in
\addtolength{\headsep}{0.25in}
\textheight 8.5 true in       % Height of text (including footnotes & figures)
\textwidth 6.0 true in        % Width of text line.
\widowpenalty=10000
\clubpenalty=10000


\pagestyle{fancy}

%\firstpageno{1}

\title{数据隐私HW3}

\author{罗浩铭\ PB21030838}


\begin{document}

\fancyhf{} % 清除所有页眉页脚
\fancyfoot[C]{\thepage} % 设置右页脚为页码
\fancyhead[l]{\footnotesize USTC Data Privacy}
% 设置右页眉为章节标题 

\renewcommand{\headrulewidth}{0pt} % 去页眉线

\begin{center}
    \textbf{\LARGE{数据隐私HW3}}\\
    \vspace{0.1cm}
    \large{罗浩铭\ PB21030838}
\end{center}

% ============1
\section{Permutation Cipher}
\subsection*{(a)}

\begin{align*}
    \pi^{-1}=
    \begin{pmatrix}
        1 & 2 & 3 & 4 & 5 & 6 & 7 & 8 \\
        4 & 1 & 6 & 2 & 7 & 3 & 8 & 5
    \end{pmatrix}^{-1}
    =
    \begin{pmatrix}
        4 & 1 & 6 & 2 & 7 & 3 & 8 & 5 \\
        1 & 2 & 3 & 4 & 5 & 6 & 7 & 8
    \end{pmatrix} \\
    =
    \begin{pmatrix}
        1 & 2 & 3 & 4 & 5 & 6 & 7 & 8 \\
        2 & 4 & 6 & 1 & 8 & 3 & 5 & 7
    \end{pmatrix}
\end{align*}

\subsection*{(b)}
原始密文如下:

TGEEMNEL NNTDROEO AAHDOETC SHAEIRLM

将其按照每8个字母一组分组,得到:

TGEEMNEL NNTDROEO AAHDOETC SHAEIRLM

将每组按照(a)中的逆置换规则进行逆置换,得到:

GENTLEME NDONOTRE ADEACHOT HERSMAIL

则解密得到明文为:
GENTLEMEN DO NOT READ EACH OTHER'S MAIL

\section{Perfect Secrecy}
\subsection*{(a)}
证明如下:

因为$|\mathcal{M}| = |\mathcal{C}| = |\mathcal{K}|$;对于每个$k \in \mathcal{K}$,其被选中的概率相等;

对于每个$m \in \mathcal{M}$与每个$c \in \mathcal{C}$,因为Latin Square中每列每个元素正好出现一次,
则有且仅有一个$k \in \mathcal{K}$使得$L(k,m)=c$,也即$e_k(m) = c$。

由此,由香农定理可得,该密码系统是完美安全的。

\subsection*{(b)}
因为该密码系统是完美安全的,且$|\mathcal{M}| = |\mathcal{C}| = |\mathcal{K}|$,由香农定理得:
\begin{itemize}
    \item $\forall k \in \mathcal{K}, P(k)=\frac{1}{|\mathcal{K}|}$
    \item $\forall m \in \mathcal{M}, c \in \mathcal{C}$,有且仅有一个$k \in \mathcal{K}$使得$e_k(m) = c$
\end{itemize}

对于每一个$m \in \mathcal{M}$,固定$m$,有以下结论:

因为$\forall c \in \mathcal{C}$,有且仅有一个$k \in \mathcal{K}$使得$e_k(m) = c$,
所以对于满足$e_k(m) = c$的密文密钥对,其构成映射$f: \mathcal{C} \to \mathcal{K}$。

因为同一个key加密$m$之后得到的密文是唯一的,所以$\forall c_1, c_2 \in \mathcal{C} s.t. c_1 \ne c_2$,其加密密钥不同,
则$f: \mathcal{C} \to \mathcal{K}$为单射。

又由于对于每个$k \in \mathcal{K}$,均存在$c \in \mathcal{C}$使得$e_k(m) = c$,所以$f: \mathcal{C} \to \mathcal{K}$为满射。

综上,$f: \mathcal{C} \to \mathcal{K}$为双射。

则$\forall c \in \mathcal{C}, P(c)=P(f(c))=\frac{1}{|\mathcal{K}|}=\frac{1}{|\mathcal{C}|}$

则对于每一个$m \in \mathcal{M}$,$\mathcal{C}$中每个密文出现的概率相等。

则$\mathcal{C}$中每个密文出现的概率相等。


\section{RSA}
\subsection*{(a)}
因为$p=101,q=113$,
所以$n=pq=11413$,$\phi(n)=(p-1)(q-1)=11200$。

由于$\phi(n)=2^6 5^2 7^1$,则$\phi(\phi(n))=11200=11200 \frac{1}{2} \frac{4}{5} \frac{6}{7} = 3840$

公钥中$n$只有一种,$e$要与$\phi(n)$互质,且$1<e<\phi(n)$,所以$e$有$\phi(\phi(n))-1=3839$种。

则公钥有$3839$种。

\subsection*{(b)}
Bob收到的密文为$C = M^e \bmod{n} = 9726^{3533} \bmod{11413} = 5761$。

因为$e=3533$,所以$ed \equiv 1 \bmod{\phi(n)}$,即$3533d \equiv 1 \bmod{11200}$,使用扩展欧几里得算法解得$d=6597$。

Bob用自己的私钥$d$将$C$其解密得到明文为$M = C^d \bmod{n} = 7624^{6597} \bmod{11413} = 9726$。

\subsection*{(c)}
对于不同的质数$p,q$,$n=pq$,$\phi(n)=(p-1)(q-1)$

则$n-\phi(n)+1=p+q$。

由韦达定理可得,$p,q$为$x^2 - (n-\phi(n)+1)x + n =0$的两个根。

则可得$p, q$的值为$\frac{1}{2} [(n-\phi(n)+1) \pm \sqrt{(n-\phi(n)+1)^2 - 4n}]$算出的两个值($p,q$必须取不同值)。

上述公式的计算只需用到$n$和$\phi(n)$的值,且计算时间复杂度为$O(1)$。因此,若知道$n$和$\phi(n)$的值,则可以在多项式时间内计算出$p,q$的值。

\section{Multi-Party Computation}
\subsection*{(a)}
\subsubsection*{(1)}
因为$p=11, q=17$,所以$n=pq=187$,$\lambda=lcm(p-1, q-1)=80$。

此处选择$g=n+1=188$,则由$L(x)=\frac{x-1}{n}$:
\begin{align*}
    \mu & =(L(g^\lambda \bmod{n^2}))^{-1} \bmod{n}    \\
        & =(L(188^{80} \bmod{187^2}))^{-1} \bmod{187} \\
        & =(L(14961))^{-1} \bmod{187}                 \\
        & = 80^{-1} \bmod{187}                        \\
        & = 180                                       \\
\end{align*}

接下来是加密过程:

由$M=175, r=83$,Bob收到的密文为$C=g^M r^n \bmod{n^2} = 188^{175} 83^{187} \bmod{187^2} = 23911$

接下来是解密过程:

解密得到的密文为:
\begin{align*}
    M' & =L(C^\lambda \bmod{n^2}) \cdot \mu \bmod{n}      \\
       & =L(23911^{80} \bmod{187^2}) \cdot 180 \bmod{187} \\
       & =L(30295) \cdot 180 \bmod{187}                   \\
       & = 162 \cdot 180 \bmod{187}                       \\
       & = 175
\end{align*}

\subsubsection*{(2)}
因为$c_1=g^{m_1} r^n \bmod{n^2}, c_2=g^{m_2} r^n \bmod{n^2}$,

所以$(c_1 \cdot c_2) \bmod{n^2} = g^{m_1+m_2} r^{2n} \bmod{n^2}$

记$r'=r^2 \bmod{n}$,由于$r$与$n$互质,则$r'$与$n$互质。

设$r'$模$n$的余数为$r_1$,则存在整数$k$使得$r' = kn + r_1$。则$r_1$与$n$互质,且$0<r_1<n$,由此$r_1$符合加密过程所需随机数的选取规则。
由此有:
\begin{align*}
      & r^{2n} \bmod{n^2}                                              \\
    = & (r')^{n} \bmod{n^2}                                            \\
    = & (kn + r_1)^{n} \bmod{n^2}                                      \\
    = & (k^n n^n + ... + n \cdot kn(r_1)^{n-1} + (r_1)^{n}) \bmod{n^2} \\
    = & (r_1)^{n} \bmod{n^2}                                           \\
\end{align*}

则$(c_1 \cdot c_2) \bmod{n^2}=g^{m_1+m_2} (r_1)^{n} \bmod{n^2}$,相当于选取了随机数$r_1$且明文为$m_1+m_2$时加密得到的密文。

由此,$Decrypt( (c_1 \cdot c_2) \bmod{n^2} ) =m_1+m_2$,同态加性质得证。

\subsection*{(b)}
定义函数$f(i)$使得$f(1)=3, f(2)=1, f(3)=2$。

对于任意共享方$P_i$,其掌握的秘密为$x_i, y_i, a_i=x_{f(i)} \bigoplus v_1, b_i=y_{f(i)} \bigoplus v_2$,
则借助以上信息,他可算得:
\begin{align*}
     & x_i \bigoplus y_i                                                              \\
     & a_i \bigoplus b_i =  x_{f(i)} \bigoplus v_1 \bigoplus y_{f(i)} \bigoplus v_2 =
    (x_{f(i)} \bigoplus y_{f(i)}) \bigoplus (v_1 \bigoplus v_2)                       \\
\end{align*}

令$z_i = x_i \bigoplus y_i, i=1,2,3$,则有:
\begin{align*}
    z_1 \bigoplus z_2 \bigoplus z_3 & = (x_1 \bigoplus y_1) \bigoplus (x_2 \bigoplus y_2) \bigoplus (x_3 \bigoplus y_3) \\
                                    & = (x_1 \bigoplus x_2 \bigoplus x_3) \bigoplus (y_1 \bigoplus y_2 \bigoplus y_3) \\
                                    & = 0 \bigoplus 0 = 0
\end{align*}

又由上述结果,每个人$P_i$无需通信就可以得到$(z_i, z_{f(i)} \bigoplus (v_1 \bigoplus v_2))$,
则无需通信每个人就可以获得$v_1 \bigoplus v_2$的秘密共享。


\section{Computational Security}
\subsection*{(a)}
Interchangeable与Indistinguishable的主要区别是:
Interchangeable要求两个函数的输出结果的各种取值的概率分布完全相同,
而Indistinguishable只要求两个函数的输出结果概率分布几乎相同,
也即概率之差是关于密钥长度$\lambda$的可忽略(negligible)函数(指其与任意$\lambda$的多项式的乘积都在$\lambda$趋于无穷时收敛至0)。

\subsection*{(b)}


\subsection*{(c)}

\end{document}